\section{Asymmetric (Public Key) Cryptography}
\label{sec::CrypCod_Asymmetric Crypto}
\begin{tabular}{|l| p{14.7cm}|}
	\hline
	Diffie-Hellman				 	& 	Choose a large public prime $p$ and a generator $g \in GF(p)$.\\
	Key Exchange					&	1. $A$ chooses a large random integer $x$, computes $g^x \mod p = c_1$ and sends the result to $B$. \\
									&	2. $B$ chooses a large random integer $y$, computes $g^y \mod p = c_2$ and sends the result to $A$. \\
									&	3. $A$ and $B$ compute each the secret key: $A: c_2^x \mod p=k $ and $B: c_1^y \mod p=k \to k=g^{xy} \mod p$.\\
									\cline{2-2}
	Security        & 	The security is based on the assumption that the discrete logarithm $x=\log_{g}(c_1)$ is hard to compute. This assumption will be definitely wrong for quantum computers!!!\\
									& Possible threat: Man-in-the-Middle attacks\\
	\hline
	ElGamal 				& 1. Choose a prime $p$ and generator $g$\\
	Key generation  & 2. Select a random secret integer $a$ with $1 \leq a \leq p-2$ and compute $A=g^a \mod p$\\
	 								& 3. The public key is: $(p, g, A)$.\\
									\cline{2-2}
	Encryption/Decryption & 1. Select a random secret integer $b$, $1 \leq b \leq p-2$ and compute $B=g^b \mod p$\\
												& 2. For message $m$, compute the cipher text $c=A^b \cdot m \mod p$\\
												& 3. The secret message consists of $B$ and $c$\\
												& 4. The cipher text $c$ can be decrypted with $B^{(p-1-a)} \cdot c$ which yields $m$\\
									\cline{2-2}
	Note & Works also in other cyclic groups (elliptic curves). ElGamal is exactly as difficult to break as Diffie-Hellman.\\
	\hline
	RSA public key 					&	$p$ and $q$ are primes (possibly large), compute $n=p\cdot q$\\
	cryptosystem key				& Encryption key is $e$, decryption key is $d$\\
	generation:						&	Choose an integer $1<e<\varphi(n)$ such that $gcd(e,\varphi(n))=1$\\
									&	Find $d$ such that $e\cdot d\equiv 1 \mod \varphi(n) \Rightarrow e\cdot d = \varphi(n) \cdot i + 1$\\
									&	public key: $(n,e)$, secret key: $(n,d), p,q, \varphi(n)$\\
									\cline{2-2}
									&	1. $A$ looks up the pulbic key $(n,e)$ of $B$\\
	encryption/decryption:			&	2. $A$ encrypt message $m \to c$: $c \equiv m^e \mod n$.\\
									&	3. $A$ sends the ciphertext $c$ to $B$.\\
									&	4. $B$ decrypts $c$ by computing $m\equiv c^d \mod n \equiv (m^{e})^d \equiv  m^{\varphi(n) \cdot i + 1} \equiv  \left(m^{\varphi(n)}\right)^i \cdot m \mod n$\\
									&	$\xrightarrow[m^{\varphi(n)} \mod n\equiv 1]{\text{little Fermat}} m \mod n \equiv m $  (use square and multiply for large $c$ and $m$)\\
									\cline{2-2}
	signature						&	$s=m^d \mod n$ only owner of $(n,d)$ can compute (sign) this.\\
									&	$m=s^e \mod n$ everyone with public $(n,e)$ can compute (verify validity) this.\\
									&	$\Rightarrow$ often only the hash $h(m)$ is signated. Attention: 
										collisions $h(m_1)=h(m_2)$.\\
									\cline{2-2}
	Security        & Factoring $n$ is probably the only way to find $d$.\\
									& Choose $p$ and $q$ properly: chosen at random and independently of each other; reject them if they only have small prime factors; n should be at least 2000 bits (3072 bits has same security level of a symmetric algorithm with key length of 128 bits)\\
	\hline 
\end{tabular}