
% Einleitung
\section{Digital Signature}
Fundamentals:
\begin{liste}
  \item	\textbf{Authentication}: Who sent the message. Is it really him?
  \item \textbf{Integrity}: Is the message still the same what the sender sent?
  \item \textbf{Non-Repudiation}: An entity cannot deny having signed it.
\end{liste}


\subsection{RSA  Signature}
The RSA signature is built as in chapter \ref{sec::CrypCod_Asymmetric Crypto} - \nameref{sec::CrypCod_Asymmetric Crypto} on page \pageref{sec::CrypCod_Asymmetric Crypto}.

\subsubsection{Attacks to RSA Signature}

\subsubsubsection{Man-in-the-Middle/ Authentication of public key}: 
Is one of the most common and dangerous attacks. The base of these attacks is that the attacker is sending you his public key and telling you it's the original key.
Then he forwards all the traffic (key exchange data) from him to the original page. 
He can now listen to all the encrypted traffic.

\subsubsubsection{No message attack}: For this the attacker chooses an arbitrary number $s$ (claims it to be RSA-signature of Alice) and computes the message $m=s^e\mod n$.
The message $m$ has yet no structure, no useful content, but it's signed by Alice.
To avoid this, the message should have some redundancy as hash value, checksum, etc.

\subsubsubsection{Common modulus attack}: $c_e=m^f \mod n \quad c_f=m^f \mod n$, known: $c_e,c_f,(n,f),(n,e)$ then $1=x \cdot e + y \cdot f$
$\rightarrow (c_e)^x \cdot (c_f)^y=(m^e)^x \cdot (m^f)^y=m^{ex+yf} \equiv m^1 \mod n$.

\subsubsubsection{Multiplicative property of RSA}:
\begin{aufzaehlung}
\item The attacker would like Alice to sign message m without showing the true content to her.
\item He chooses any $m_1$ with $gcd(m_1,n)=1$
\item He computes $m_2 = m \cdot m_1^{-1} \mod n$
\item He asks Alice to sign $m_1$,$m_2$ and receives $s_1$,$s_2$
\item $s=s_1 \cdot s_2 \mod n \equiv (m_1\cdot m_2)^d \mod n \equiv m^d \mod n$ is the valid signature of m
\end{aufzaehlung}
Some redundancy in the messages can avoid this attacks.
Also, Alice should sign some ``random'' messages.

\subsection{Hash Functions}
A hash function maps binary strings of arbitrary length to binary strings of some fixed length, called hash-values:\\ $h: \{0,1\}^* \mapsto \{0,1\}^n$\\
Cryptographic hash functions need to fulfil three properties:
\begin{liste}
	\item [\-] \textbf{Preimage Resistance:} For a given hash value, it is difficult to find a corresponding input string; $h(m)$ is a one-way function
	\item [\-] \textbf{Second-Preimage Resistance:} Given an input $m_1$, it is difficult to find a second input $m_2$ such that $h(m_1)=h(m_2)$; also called weak collision resistance
	\item [\-] \textbf{Collision Resistance:} It is difficult to find two different input strings $m_1$ and $m_2$ such that $h(m_1)=h(m_2)$; also called strong collision resistance
\end{liste}
Advantages:
\begin{liste}
	\item Only short hash value $h(m)$ must be signed, instead of possibly very long message $m$.
	\item No-Message-Attack doesn't work, since $h(m)$ is a one-way function.
	\item Exploiting the multiplicative property is not possible, since $h(m)$ is a one-way function.
	\item The signed message $m$ cannot be replaced by another message $m^*$, since they would be a collision of the hash function.
\end{liste}
Examples:
\begin{liste}
	\item Secure Hash Algorithm (SHA)
	\begin{liste}
		\item SHA-1: 160-bit hash value, no longer considered secure!
		\item SHA-2: 224,256,384,512-bit hash value
	\end{liste}
	\item Message-Digest algorithm 5 (MD-5)
	\begin{liste}
		\item 128-bit hash value, no longer considered secure!
	\end{liste}
	\item RACE Integrity Primitives Evaluation Message Digest (RIPEMD)
	\begin{liste}
		\item RIPEMD: no longer considered secure!
		\item RIPEMD-160, RIPEMD-320: considered secure
	\end{liste}
\end{liste}

\subsection{MAC (Message Authentication Codes)}
Use parametrized hash function $\to$ $h_k: \sum^* \to \sum^4, x \to g(x) \oplus k$. $k$ is in this example a 4bit secret key.\\
Example: $A$ would like to send a message to $B$, which isn't confidential, but sure, that the message wasn't changed during transmission.\\
\begin{tabular}{l}
	- $A$ and $B$ exchange a secret key $k$ (strictly spoken, this exchange should be authenticated as well).\\
	- $A$ sends the message $m$ together with the MAC value $y=h_k(m)$ to $B$.\\
	- $B$ computes $y'=h_k(m')$ with the received message $m'$. $B$ accepts the message, if $y'=y$.
\end{tabular}

\subsection{DSA (Digital Signature Algorithm)}
DSA is a more efficient variant of the ElGamal signature algorithm.\\
Generate a public key $(\bm {p,q,g,y})$ and the secret key $\bm x$:
\begin{aufzaehlung}
\item Select a prime number $\bm q$ with exactly 160 binary digits $(2^{159}< \bm q < 2^{160})$
\item Choose $t; 0 \leq t \leq 8$ and select a prime number $\bm p$ where $2^{511+64t}< \bm p < 2^{512+64t}$ and $\bm q$ divides $(\bm p-1)$. The length of $\bm q$ is between 512 and 1024 binary digits and is a multiple of 64.
\item Choose a number $1< h<\bm p$ and compute $\bm g = h^{(\bm p-1) / \bm q} \mod \bm p$ if g=1 repeat this step.
($\bm g$ is a generator of the subgroup of order $q \mod p$, such that $1<g<p$)
\item Select a random integer $1 \leq \bm x \leq \bm q-1$
\item Compute $\bm y=\bm g^{\bm x} \mod \bm p$
\end{aufzaehlung}
Sign a public, collision-free hash function $h(m)$ and return signature $(\bm {r,s})$:
\begin{aufzaehlung}
\item Select a random integer $0< k<\bm q$
\item $\bm r = (\bm g^k \mod \bm p) \mod \bm q$
\item $\bm s=(k^{-1} \cdot (h(m) + \bm x\cdot \bm r))\mod \bm q$
\end{aufzaehlung}
Verification with public key $(\bm {p,q,g,y})$ and signature $(\bm {r,s})$:
\begin{aufzaehlung}
\item Verify that $0<\bm r<\bm q$ and$0<\bm s<\bm q$
\item $w=(\bm s^{-1})\mod\bm q$
\item $u_1=( w\cdot h(m)) \mod \bm q$
\item $u_2=( w \cdot \bm r) \mod\bm q$
\item $\bm v=((\bm g^{u_1} \cdot \bm y^{u_2})\mod \bm  p) \mod \bm q$
\item if $\bm v= \bm r \Rightarrow$ Accept it!
\end{aufzaehlung}



\subsection{ElGamal}
Find primitive roots in a given field $\mathbb{F}_p$.
\begin{aufzaehlung}
\item Factor $p-1=p_1^{e1} \cdot p_2^{e2} \ldots p_k^{ek}$
\item Choose a random element $\alpha$ in $\mathbb{F}_p$ with $2 \leq \alpha \leq p-1$
\item For $i$ from 1 to $k$:
Compute $b=\alpha ^{\frac{p-1}{p_i}} \mod p \Rightarrow$ if $b=1$, go to step before.
\item Return $\alpha$
\end{aufzaehlung}

Generate a key $a$ in $\mathbb{F}_p$.
\begin{aufzaehlung}
\item Chosse a prime $p$ and determine a primitive root $\alpha$
\item Select a random secret integer $a$ with $1 \leq a \leq p-2$
\item Compute $y=\alpha^a \mod p$
\item The public key consists of $p$, $\alpha$ and y. The secret key is $a$.
\end{aufzaehlung}

ElGamal Signature Generation, requirement: $(\mathbb{F}_p, a, \alpha)$
\begin{aufzaehlung}
  	\item  Select a random secret integer $k$, $1 \leq k \leq gcd(k,p-1)=1$ 
  	\item  Compute $r=\alpha^k \mod p$ 
  	\item  Compute $k^{-1} \mod (p-1) \to$ (with extended Euclid)
  	\item  Compute $s=k^{-1} \cdot \lbrack h(m) - a \cdot r \rbrack \mod (p-1)$
  	\item  The signature consists of r and s.
\end{aufzaehlung}

Verification of the signature $(r,s)$
\begin{aufzaehlung}
  \item   Obtain the public key $(p,\alpha,y)$
  \item   Verify that $1 \leq r \leq p-1$. If not, then reject the signature.
  \item   Compute $v_1=y^r \cdot r^s \mod p$
  \item   Compute $h(m)$ and $v_2=\alpha^{h(m)} \mod p$
  \item   Accept the signature $(r,s)$ if and only if $v_1=v_2$
\end{aufzaehlung}

\textbf{Possible Attacks:} If an attacker knows the randomly chosen integer $k$, he could multiply $s$ with $k$ and would then be able to compute the secret key $a$.


\subsection{Public Key Infrastructure - PKI}
The \em Public Key Infrastructure \em is used to verify which certificate a user can trust and which not.\\
\subsubsubsection{Direct Trust}: 
In this model, a user just trusts people which he met and whose public keys he got. \\
\subsubsubsection{Hierachical Trust}: A \em meta introducer \em (root CA) (CA=Certification Authority) certifies other trusted introducers.
The public key of the meta introducer is already installed on the web browser, operation system, etc. \\
\subsubsubsection{Web of Trust}: It's based on references. You have some user you know and \em trust \em and you know that they check the new references quite well.
So you'll accept a ceritficate which he accepted also.\\
Other users: you may accept their certificate but you trust only \em marginal \em.
So a new certificate needs to be trusted by many such people until you trust it.
Then there are users you don't trust (\em untrusted user \em).

